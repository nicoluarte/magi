% Created 2019-07-29 Mon 15:25
% Intended LaTeX compiler: pdflatex
\documentclass[11pt]{article}
\usepackage[utf8]{inputenc}
\usepackage[T1]{fontenc}
\usepackage{graphicx}
\usepackage{grffile}
\usepackage{longtable}
\usepackage{wrapfig}
\usepackage{rotating}
\usepackage[normalem]{ulem}
\usepackage{amsmath}
\usepackage{textcomp}
\usepackage{amssymb}
\usepackage{capt-of}
\usepackage{hyperref}
\usepackage{setspace}
\onehalfspacing
\usepackage[round]{natbib}
\usepackage[margin=1.2in]{geometry}
\renewcommand\refname{Referencias}
\author{Luis Nicolás Luarte Rodríguez}
\date{}
\title{Declaración de propósitos}
\hypersetup{
 pdfauthor={Luis Nicolás Luarte Rodríguez},
 pdftitle={Declaración de propósitos},
 pdfkeywords={},
 pdfsubject={},
 pdfcreator={Emacs 25.2.2 (Org mode 9.2.4)}, 
 pdflang={English}}
\begin{document}

\maketitle
\section{Motivación personal para el programa de Doctorado}
\label{sec:org18057e2}
El cómo buscamos objetos, información, recompensas, alimentos, etc. Ha
sido lo que ha inspirado en mayor medida mi interés en la
neurociencia. A lo largo de mi vida he sentido profunda intriga en
cómo los humanos buscan en el espacio de posibilidades, para tomar una
decisión, para evocar una memoria en particular o bien simplemente
para organizar cualquier comportamiento relativamente complejo. Investigar
sobre los mecanismos subyacentes a ese fenómeno ha sido increíblemente
enriquecedor debido al fuerte componente interdisciplinar del campo.
Esto me ha llevado a generar un profundo interés en seguir
desarrollando mi carrera en neurociencia, ya que, creo, el lograr
entender ese aparentemente simple mecanismo de decisión en condiciones
de información incompleta, puede, eventualmente, ser de gran utilidad
para la comprensión tanto de procesos de memoria y aprendizaje cómo de
ciertas patologías. Con la oportunidad del programa de Doctorado
espero contribuir a la investigación del aprendizaje y memoria.

Cómo parte de mi formación en el programa de Magíster en Neurociencias
Social de la Universidad Diego Portales, investigue, cómo parte de un
artículo de revisión, las raíces evolutivas de la búsqueda semántica
(recuperación de memorias en tareas de evocación). Una de las
principales conclusiones fue que, aunque solo en grado tentativo,
parece existir un mecanismo compartido entre la búsqueda semántica y
el forrajeo ('foraging', el comportamiento de búsqueda de alimento),
teniendo este último patrones relativamente marcados, que se extienden
a lo largo de miles de años, así cómo a través de múltiples
especies. La posibilidad de que un mecanismo tan ubicuo, responsable
del comportamiento de desplazamiento en la búsqueda de alimentos, pueda estar
relacionado por exaptación a un proceso fundamental de la memoria abre una
posibilidad de establecer un mapeo evolutivo al menos a este proceso de memoria.

Deseoso de aprender más sobre este posible vínculo entre forrajeo y
memoria, me adentre en las principales áreas aledañas de conocimiento,
tales como ecología, aprendizaje por reforzamiento ('reinforcement
learning') y otros modelos computacionales. Por la alta carga de modelos
estadísticos en las áreas mencionadas, me apunté para un programa de
diplomado en ciencia de datos de la Universidad Católica de
Chile. Además de este programa he realizado aprendizaje autónomo en
cursos en línea, con el fin de contar con todas las herramientas
técnicas que son demandadas para el área.

Adicional a los programas mencionados anteriormente, desde julio del año
2018, me encuentro participando como investigador en un proyecto
FONDECYT conjunto entre la escuela de Arquitectura y Psicología de la
Universidad Diego Portales. El tema central de esta investigación es el
estudio de la percepción de peatones en diferentes ambientes
urbanos. Si bien el tema no está relacionado directamente con el área
de interés, mi rol ha consistido en utilización de técnicas de visión
de máquina ('machine vision') y procesamiento de datos tanto para
'Eye-tracker' cómo para análisis de frecuencia de objetos. Lo
anterior, adicionado a el aprendizaje de diversos lenguajes de
programación (MATLAB, Python, R, Bash), me ha permitido desarrollar
herramientas que son útiles en la investigación en general cómo
específicamente para el área de mi interés.

\section{Formulación tópico de interés}
\label{sec:org8731e1d}
Mi tópico de interés reside en el estudio de la memoria, específicamente, la
búsqueda semántica, esto es, las estrategias o patrones utilizados al momento de
recuperar un contenido de memoria, a través del estudio de patrones
evolutivamente semejantes tales como el de 'forrajeo'.

 Las memorias semánticas han sido pensadas, teóricamente, cómo elementos
pertenecientes a cierto 'espacio' constituido por la similitud en significado
\citep{lundProducingHighdimensionalSemantic1996}. Así, se ha propuesto una
'distancia' entre los distintos contenidos semánticos
\citep{montezRoleSemanticClustering2015} y por lo tanto la posibilidad de
'navegar' entre dichos contenidos. Considerando lo anterior, es esperable que a
lo largo de la evolución se hayan generado estrategias para acceder, de manera
útil e eficiente, a dichos contenidos. Las estrategias de búsqueda para acceder
a los contenidos semánticos han sido relacionadas a aquellas del forrajeo
\citep{ForagingSemanticFields,hillsAnimalForagingEvolution2006,hillsOptimalForagingSemantic2012},
en tanto las estrategias, para ambos casos, deben lidiar con el dilema de
explorar/explotar.

Se ha observado que los 'algoritmos' utilizados en el forrajeo, pueden proveer
de soluciones óptimas para dicho dilema
\citep{bartumeusAnimalSearchStrategies2005a}, lo cuál aplicaría, igualmente, para
estrategias en espacios semánticos \citep{montezRoleSemanticClustering2015}. De
esta manera se puede observar una conexión entre un mecanismo evolutivamente
antiguo (forrajeo) y la búsqueda semántica. Permitiendo un enfoque evolutivo
comprensivo al estudio de la memoria, correspondiente al tema de mi interés.

El cómo se realiza la búsqueda en espacios semánticos es de
fundamental importancia, ya que es un espacio que está en activa
búsqueda durante la comprensión y producción de lenguaje, entre otras
\citep{montezRoleSemanticClustering2015}, por lo mismo, el alcance de su
importancia para casi cualquier actividad cognitiva es de gran tamaño,
pudiendo afectar de manera importante el comportamiento ante múltiples y
diferentes tareas.

\section{Objetivos a corto plazo}
\label{sec:org2c35155}
Uno de los principales tópicos de discusión en el área de búsqueda
semántica es la organización y el tipo de la relaciones que conforman
el espacio semántico \citep{lundProducingHighdimensionalSemantic1996}. Así, Uno de los primeros
objetivos de investigación sería poder generar configuraciones
experimentales que permitiesen determinar, principalmente, (a) efecto
del contexto en las relaciones entre contenidos semánticos
\citep{schillerMemorySpaceUnderstanding2015} y (b) si el tipo de búsqueda es más
verosímil para contenidos encadenados de manera asociativa o categórica
\citep{hillsOptimalForagingSemantic2012}. 

Cómo segundo objetivo a corto plazo, de manera experimental, modelar el
comportamiento en tareas de evocación de memoria, a modo de sugerir posibles
mecanismos generadores del comportamiento de búsqueda semántica. Los modelos mas
relevantes son (a) aquellos basados en reglas
\citep{charnovOptimalForagingMarginal1976}, (b) modelos aleatorios simples
\citep{thompsonWalkingWikipediaScalefree2014} y (c) modelos aleatorios complejos
cómo discutido en \citep{benhamouHowManyAnimals2007}
\section{Objetivos a largo plazo}
\label{sec:orgbe60ffa}
Los objetivos a corto plazo están relacionados, principalmente, con el
modelamiento de comportamiento en tareas de búsqueda semántica. Por otro lado,
los objetivos a largo plazo buscarían conectar dichos modelos a sus estructuras
cerebrales (u otras) subyacentes. Uno de los candidatos parece ser el 'Locus
coeruleus' \citep{kaneIncreasedLocusCoeruleus2017}, corteza medial pre-frontal
ventromedial \citep{kollingNeuralMechanismsForaging2012}, corteza cingulada
anterior \citep{shenhavAnteriorCingulateEngagement2014}, entre otras. Si bien la
función del 'Locus Coeruleus' es extendida en el sistema de arousal, se observa
evidencia de participación significativa en comportamiento de
exploración-explotación relacionados al forrajeo
\citep{aston-jonesAdaptiveGainRole2005}, además la relativa facilidad en medición,
en conjunto con otras técnicas como electro-encefalograma, ha permitido
encontrar relación entre este y el forrajeo en situaciones experimentales
\citep{slanziCombiningEyeTracking2017}. Esto implicaría la combinación de modelos
de comportamiento y registro de medidas fisiológicas, tales como diámetro de
pupila. El tener un modelo de comportamiento, en el marco de explorar/explotar,
permitiría eventualmente, tener una idea de que estrategia se esta ocupando en
cada decisión y por ende evaluar la contribución de las diferentes estructuras
cerebrales a esto.

Adicionalmente, se encuentra dentro de mis objetivos a largo plazo la docencia
en el área de neurociencias, principalmente buscando ser un aporte para la
promoción de la ciencia experimental en Psicología, la cual tradicionalmente
tiene un espacio muy reducido en el currículo de pre-grado, desaprovechando un
campo muy fértil en investigación.
\begin{Latex}
\pagebreak
\end{Latex}
\bibliographystyle{apa}
\bibliography{ref}
\end{document}
