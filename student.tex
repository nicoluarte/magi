% Created 2019-09-09 Mon 11:34
% Intended LaTeX compiler: pdflatex
\documentclass[11pt]{article}
\usepackage[utf8]{inputenc}
\usepackage[T1]{fontenc}
\usepackage{graphicx}
\usepackage{grffile}
\usepackage{longtable}
\usepackage{wrapfig}
\usepackage{rotating}
\usepackage[normalem]{ulem}
\usepackage{amsmath}
\usepackage{textcomp}
\usepackage{amssymb}
\usepackage{capt-of}
\usepackage{hyperref}
\usepackage{setspace}
\onehalfspacing
\usepackage[round]{natbib}
\usepackage[margin=0.95in]{geometry}
\renewcommand\refname{Referencias}
\author{Luis Nicolás Luarte Rodríguez}
\date{}
\title{Declaración de propósitos}
\hypersetup{
 pdfauthor={Luis Nicolás Luarte Rodríguez},
 pdftitle={Declaración de propósitos},
 pdfkeywords={},
 pdfsubject={},
 pdfcreator={Emacs 25.2.2 (Org mode 9.2.4)}, 
 pdflang={English}}
\begin{document}

\maketitle
\section*{Motivación personal para el programa de Doctorado}
\label{sec:orge6072d9}
El cómo buscamos objetos, información, recompensas, alimentos, etc. Ha sido lo
que ha inspirado en mayor medida mi interés en la neurociencia. A lo largo de mi
vida he sentido profunda intriga en cómo los humanos buscan en el espacio de
posibilidades, para tomar una decisión, para evocar una memoria en particular o
bien simplemente para organizar cualquier comportamiento relativamente complejo.
Investigar sobre los mecanismos subyacentes a ese fenómeno ha sido
increíblemente enriquecedor debido al fuerte componente interdisciplinar del
campo. Esto me ha llevado a generar un profundo interés en seguir desarrollando
mi carrera en neurociencia, ya que, creo, el realizar avances en la comprensión
de los mecanismos de decisión y búsqueda en condiciones de información
incompleta, puede, eventualmente, ser de gran utilidad para comprender los
procesos de memoria y aprendizaje, en tanto estos últimos se pueden comprender
como un proceso de búsqueda de contenidos en un espacio mental. Con la
oportunidad del programa de Doctorado espero contribuir a la investigación del
aprendizaje y memoria.

En el programa de Magíster en Neurociencias Social de la Universidad Diego
Portales, investigue, como parte de un artículo de revisión, las raíces
evolutivas de la búsqueda semántica (recuperación de memorias en tareas de
evocación). Una de las principales conclusiones fue que, aunque sólo en grado
tentativo, parece existir un mecanismo compartido entre la búsqueda semántica y
el forrajeo ('foraging', el comportamiento de búsqueda de alimento), teniendo
este último patrones relativamente marcados, que se extienden a lo largo de
miles de años, así como a través de múltiples especies. La posibilidad de que un
mecanismo tan ubicuo, responsable del comportamiento de desplazamiento en la
búsqueda de alimentos, pueda estar relacionado por exaptación a un proceso
fundamental de la memoria abre una posibilidad de establecer un mapeo evolutivo
al menos a este proceso de memoria.

En la investigación mencionada anteriormente, hallé interesantes vínculos
interdisciplinares, por ejemplo, entre neurociencia de la toma de decisiones y
etología \citep{mobbsForagingFoundationsDecision2018a}, lo que me permitió
postular la toma de decisiones bajo modelos de forrajeo, tales como los
propuestos por el 'Marginal Value Theorem'
\citep{charnovOptimalForagingMarginal1976}. Además, me adentre en modelos
computacionales de la toma de decisiones
\citep{aston-jonesINTEGRATIVETHEORYLOCUS2005a}, y finalmente relacionar los
diversos modelos de toma de decisiones secuenciales con la búsqueda de
contenidos semánticos \citep{hillsForagingSemanticFields2015}. Siendo lo
anteriores tópicos de mi particular interés.

\section*{Formulación del tópico de interés}
\label{sec:org97513b9}
Mi tópico de interés principal reside en el estudio de la memoria,
específicamente, la búsqueda semántica, esto es, las estrategias o patrones
utilizados al momento de recuperar un contenido de memoria.

 Las memorias semánticas han sido pensadas, teóricamente, como elementos
pertenecientes a cierto 'espacio' constituido por la similitud en significado
\citep{lundProducingHighdimensionalSemantic1996}. Así, se ha propuesto una
'distancia' entre los distintos contenidos semánticos
\citep{montezRoleSemanticClustering2015}, y por lo tanto, la posibilidad de
'navegar' entre dichos contenidos. Considerando lo anterior, es esperable que a
lo largo de la evolución se hayan generado estrategias para acceder, de manera
útil e eficiente, a dichos contenidos. Las estrategias de búsqueda para acceder
a los contenidos semánticos han sido relacionadas a aquellas del forrajeo
\citep{hillsAnimalForagingEvolution2006,hillsOptimalForagingSemantic2012},
en tanto las estrategias, para ambos casos, deben lidiar con el dilema de
explorar/explotar \citep{berger-talExplorationExploitationDilemmaMultidisciplinary2014}.

Se ha observado que los 'algoritmos' utilizados en el forrajeo, pueden proveer
de soluciones óptimas para dicho dilema
\citep{bartumeusAnimalSearchStrategies2005a}, lo cual aplicaría, igualmente, para
estrategias en espacios semánticos \citep{montezRoleSemanticClustering2015}. De
esta manera se puede observar una conexión entre un mecanismo evolutivamente
antiguo (forrajeo) y la búsqueda semántica. Permitiendo un enfoque evolutivo
comprensivo al estudio de la memoria, correspondiente al tema de mi interés.

El cómo se realiza la búsqueda en espacios semánticos es de
fundamental importancia, ya que es un espacio que está en activa
búsqueda durante la comprensión y producción de lenguaje, entre otras
\citep{montezRoleSemanticClustering2015}, por lo mismo, el alcance de su
importancia para casi cualquier actividad cognitiva es de gran tamaño,
pudiendo afectar de manera importante el comportamiento ante múltiples y
diferentes tareas.
\section*{Objetivos a corto plazo}
\label{sec:orgb0a0ba6}
Uno de los principales tópicos de discusión en el área de búsqueda
semántica es la organización y el tipo de la relaciones que conforman
el espacio semántico \citep{lundProducingHighdimensionalSemantic1996}. Así, Uno de los primeros
objetivos de investigación sería poder generar tareas
experimentales que permitan determinar, principalmente, (a) efecto
del contexto en las relaciones entre contenidos semánticos
\citep{schillerMemorySpaceUnderstanding2015} y (b) si el tipo de búsqueda es más
verosímil para contenidos encadenados de manera asociativa o categórica
\citep{hillsOptimalForagingSemantic2012}. 

Cómo segundo objetivo a corto plazo, de manera experimental, modelar el
comportamiento en tareas de evocación de memoria, a modo de sugerir posibles
mecanismos generadores del comportamiento de búsqueda semántica. Los modelos más
relevantes son (a) aquellos basados en reglas
\citep{charnovOptimalForagingMarginal1976}, (b) modelos aleatorios simples
\citep{thompsonWalkingWikipediaScalefree2014} y (c) modelos aleatorios complejos
cómo discutido en \citep{benhamouHowManyAnimals2007}
\section*{Objetivos a largo plazo}
\label{sec:org270b4e9}
Los objetivos a corto plazo están relacionados, principalmente, con el
modelamiento de comportamiento en tareas de búsqueda semántica. Por otro lado,
los objetivos a largo plazo buscarían conectar dichos modelos a sus estructuras
cerebrales (u otras) subyacentes. Uno de los candidatos parece ser el 'Locus
coeruleus' \citep{kaneIncreasedLocusCoeruleus2017}, corteza medial pre-frontal
ventromedial \citep{kollingNeuralMechanismsForaging2012}, corteza cingulada
anterior \citep{shenhavAnteriorCingulateEngagement2014}, entre otras. Si bien la
función del 'Locus Coeruleus' es extendida en el sistema de arousal, se observa
evidencia de participación significativa en comportamiento de
exploración-explotación relacionados al forrajeo
\citep{aston-jonesAdaptiveGainRole2005}, además la relativa facilidad en medición,
en conjunto con otras técnicas como electro-encefalograma, ha permitido
encontrar relación entre este y el forrajeo en situaciones experimentales
\citep{slanziCombiningEyeTracking2017}. Esto implicaría la combinación de modelos
de comportamiento y registro de medidas fisiológicas, tales como diámetro de
pupila. El tener un modelo de comportamiento, en el marco de explorar/explotar,
permitiría eventualmente, tener una idea de qué estrategia se está ocupando en
cada decisión y por ende evaluar la contribución de las diferentes estructuras
cerebrales a esto.

Adicionalmente, se encuentra dentro de mis objetivos a largo plazo la docencia
en el área de neurociencias, principalmente buscando ser un aporte para la
promoción de la ciencia experimental en Psicología, la cual tradicionalmente
tiene un espacio muy reducido en el currículo de pre-grado, desaprovechando un
campo muy fértil en investigación.
\section*{Conocimiento relevante}
\label{sec:orgc572141}
Por la alta carga de modelos estadísticos en las áreas de interés mencionadas,
me apunté para un programa de diplomado en ciencia de datos de la Universidad
Católica de Chile, dónde he aprendido fundamentos computacionales usados en
teorías de toma de decisiones, además de herramientas estadísticas necesarias.
Adicionalmente, he aprendido teoría de aprendizaje por refuerzo
('reinforcement learning') \citep{suttonReinforcementLearningDirect1992}, lo que
aporta una base para la comprensión de muchos de los modelos mencionados con
anterioridad.

Adicional a los programas mencionados anteriormente, desde julio del año 2018,
me encuentro participando como investigador en un proyecto FONDECYT conjunto
entre la escuela de Arquitectura y Psicología de la Universidad Diego
Portales. El tema central de esta investigación es el estudio de la percepción
de peatones en diferentes ambientes urbanos. Si bien el tema no está
relacionado directamente con el área de interés, mi rol ha consistido en
ajuste de modelos estadísticos, utilización de técnicas de visión de máquina
('machine vision') y procesamiento de datos tanto para 'Eye-tracker' como para
análisis de frecuencia de objetos. Lo anterior, adicionado a el aprendizaje de
diversos lenguajes de programación (MATLAB, Python, R, Bash), me ha permitido
desarrollar herramientas que son útiles en la investigación en general como
específicamente para el área de mi interés.
\begin{Latex}
\pagebreak
\end{Latex}
\bibliographystyle{apa}
\bibliography{ref}
\end{document}
