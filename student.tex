% Created 2019-07-14 Sun 23:01
% Intended LaTeX compiler: pdflatex
\documentclass[11pt]{article}
\usepackage[utf8]{inputenc}
\usepackage[T1]{fontenc}
\usepackage{graphicx}
\usepackage{grffile}
\usepackage{longtable}
\usepackage{wrapfig}
\usepackage{rotating}
\usepackage[normalem]{ulem}
\usepackage{amsmath}
\usepackage{textcomp}
\usepackage{amssymb}
\usepackage{capt-of}
\usepackage{hyperref}
\usepackage{setspace}
\onehalfspacing
\usepackage[round]{natbib}
\author{Luis Nicolás Luarte Rodríguez}
\date{}
\title{Declaración de propósitos}
\hypersetup{
 pdfauthor={Luis Nicolás Luarte Rodríguez},
 pdftitle={Declaración de propósitos},
 pdfkeywords={},
 pdfsubject={},
 pdfcreator={Emacs 25.2.2 (Org mode 9.2.4)}, 
 pdflang={English}}
\begin{document}

\maketitle
\section{Motivación I}
\label{sec:org8c3ca2f}
El cómo buscamos objetos, información, recompensas, alimentos, etc. Ha
sido lo que ha inspirado en mayor medida mi interés en la
neurociencia. A lo largo de mi vida he sentido profunda intriga en
cómo los humanos buscan en el espacio de posibilidades, para tomar una
decisión, para evocar una memoria en particular o bien simplemente
para organizar cualquier comportamiento relativamente complejo, esto
es, sin tener de antemano consideradas todas las posibilidades y aún
pese a eso tener un buen desempeño en múltiples tareas. Investigar
sobre los mecanismo subyacentes a ese fenómeno ha sido increíblemente
enriquecedor debido al fuerte componente interdisciplinar que subyace al
campo. Esto me ha llevado a generar un profunda interés en seguir
desarrollando carrera en neurociencia, ya que, creo, el lograr
entender ese aparentemente simple mecanismo de decisión en condiciones
de información incompleta, puede eventualmente, ser de gran utilidad
para la comprensión tanto de procesos de memoria y aprendizaje cómo de
ciertas patologías. Con la oportunidad del programa de Doctorado
espero contribuir a la investigación del aprendizaje y memoria.
\section{Background}
\label{sec:org9334018}
Cómo parte de mi formación en el programa de Magíster en Neurociencias
Social de la Universidad Diego Portales, investigue, cómo parte de un
artículo de revisión, las raíces evolutivas de la búsqueda semántica
(recuperación de memorias en tareas de evocación). Una de las
principales conclusiones fue que, aunque solo en grado tentativo,
parace existir un mecanismo compartido entre la búsqueda semántica y
el forrajeo ('foraging', el comportamiento de búsqueda de alimento),
teniendo este último patrones relativamente marcados que se extienden
a lo largo de miles de años, así cómo a través de multiples
especies. La posibilidad de que un mecanismo tan ubiquo, responsable
del comportamiento motil en la búsqueda de alimentos, pueda estar
relacionado por exaptación a un proceso fundamental de la memoria. Lo
que abre una posibilidad de establecer un mapeo evolutivo al menos a
este proceso de memoria.
\section{Motivación II}
\label{sec:org39a96a3}
Deseoso de aprender más sobre este posible vínculo entre forrajeo y
memoria, me adentre en las principales áreas aledañas de conocimiento,
tales como ecología, aprendizaje por reforzamiento ('reinforcement
learning') y modelos computacionales. Por la alta carga de modelos
estadísticos de las áreas mencionadas, me apunté para un programa de
diplomado en ciencia de datos de la Universidad Católica de
Chile. Además de este programa he realizado aprendizaje autonomo en
cursos en línea, con el fin de contar con todas las herramientas
técnicas que son demandadas para el área.
\section{Motivación III (investigación)}
\label{sec:org64c9ec5}
Adicional los programas mencionados anteriormente, desde julio del año
2018, me encuentro participando como investigador en un proyecto
FONDECYT conjunto entre la escuela de Arquitectura y Psicología de la
Univerdad Diego Portales. El tema central de esta investigación es el
estudio de la percepción de peatones en diferentes ámbientes
urbanos. Si bien, el tema no está relacionado directamente con el área
de interés, mi rol ha consistido en utilización de técnicas de visión
de máquina ('machine vision') y procesamiento de datos tanto para
'Eye-tracker' cómo para análisis de frecuencia de objetos. Lo
anterior, adicionado, a el aprendizaje de diversos lenguajes de
programación (MATLAB, Python, R, Bash) me ha permitido desarrollar
herramientas que son útiles en la investigación en general cómo
especificamente para el área de mi interés.
\section{Formulación tópico de interés}
\label{sec:orgf423dae}
\subsection{Introducción}
\label{sec:org56f7ab9}
Mi tópico de interés reside en el estudio de la memoria,
específicamente la búsqueda semántica. Las memorias semánticas han
sido pensadas, teóricamente, cómo elementos pertenecientes a cierto
'espacio' que correlaciona con la similitud en significado (Lund
1996). Así se ha propuesto una 'distancia' entre los distintos
contenidos semánticos (Montez 2015), considerando aquello es esperable
que a lo largo de la evolución se hayan generado estrategias para
acceder, de manera útil e eficiente, a dichos contenidos. Las
estrategias de búsqueda para acceder a los contenidos semánticos han
sido relacionadas a aquellas del forrajeo (Hills 2015, 2008, 2006,
2009, Abbott 2015). Más aún, se ha propuesto que dichos contenidos se
agrupan en 'parches' (Abbot 2015), y que la búsqueda a través de ellos
puede ser descrita por caminatas aleatorias (Hills 2015), a la vez que
siguen comportamiento basados en reglas similares a los del forrajeo
(Davelaar 2015)

Dado que la búsqueda semántica es un comportamiento orientado a
objetivos, se puede conceptualizar cómo un comportamiento orientado a
la obtención de recompensas en un espacio de múltiples
posibilidades. Por lo anterior, puede ser estudiado desde el dilema de
exploración-explotación, dilema extensamente estudiado en la tarea
'n-armed bandit' (Macready 1998, Vermorel 2005). Ha sido propuesto que
los 'algoritmos' utilizados en el forrajeo, pueden proveer de
soluciones óptimas para dicho dilema (Viswanathan, Bartumeus 2005), lo
cuál aplicaria, igualmente, para estrategias en espacios semánticos
(Abbot 2015, Montez 2015). De esta manera se puede observar una
conexión entre un mecanismo evolutivamente antiguo (forrajeo) y el
proceso de acceso en la memoria. Lo cúal permitiria un enfoque
evolutivo comprensivo al estudio de la memoria.
\subsection{Relevancia}
\label{sec:org2bf5f1d}
El cómo se realiza la búsqueda en espacios semánticos es de
fundamental importacia, ya que es un espacio que está en activa
búsqueda durante la comprensión y producción de lenguaje, entre otras
(\url{https://doi.org/10.1111/cogs.12249}), por lo mismo el alcance de su
importancia para casi cualquier actividad cognitiva es de gran tamaño,
puediendo afectar de manera importante el comportamiento ante múltiples y
diferentes tareas.
\section{Objetivos a corto plazo}
\label{sec:orga3951c7}
Uno de los principales tópicos de discusión en el área de búsqueda
semántica es la organización y el tipo de la relaciones que conforman
el espacio semántico (Lund \& Burgess 1996). Uno de los primeros
objetivos de investigación sería poder generar configuraciones
experimentales que permitiesen determinar, principalmente, (a) efecto
del contexto en las relaciones entre contenidos semánticos
\citep{schillerMemorySpaceUnderstanding2015} y (b) si el tipo de búsqueda es más
verosímil para contenidos encadenados de manera asociativa o categórica
\citep{hillsOptimalForagingSemantic2012}. 

Cómo segundo objetivo a corto plazo, de manera experimental, ajustar modelos en
tareas de evocación de memoria, a modo de sugerir posibles mecanismos
generadores del comportamiento de búsqueda semántica. Los modelos mas
relevantes son (a) aquellos basados en reglas \citep{charnovOptimalForagingMarginal1976}, (b)
modelos aleatorios simples \citep{thompsonWalkingWikipediaScalefree2014} y (c) modelos aleatorios
complejos cómo discutido en \citep{benhamouHowManyAnimals2007}  
\section{Objetivos a largo plazo}
\label{sec:org78c6fe1}
\subsection{Hipótesis sobre mecanismos subyacentes}
\label{sec:orgda89443}
Los objetivos a corto plazo están relacionados, principalmente, con estudio de
comportamiento e inferencia de posibles mecanismos generadores. Por otra parte,
los objetivos a largo plazo buscarían conectar dichos modelos a sus estructuras
cerebrales (u otras) subyacentes. Uno de los candidatos parece ser el 'Locus
coeruleus' \citep{kaneIncreasedLocusCoeruleus2017}, corteza medial pre-frontal
ventromedial \citep{kollingNeuralMechanismsForaging2012}, corteza cingulada
anterior \citep{shenhavAnteriorCingulateEngagement2014}, entre otras. Si bien la
función del 'Locus coeruleus' es extendida en el sistema de arousal, se observa
evidencia de participación significativa en comportamiento de
exploración-explotación relacionados al forrajeo
\citep{aston-jonesAdaptiveGainRole2005}, además la relativa facilidad en medición,
en conjunto con otras técnicas como electro-encefalograma, ha permitido
encontrar relación entre este y el forrajeo en situaciones experimentales
\citep{slanziCombiningEyeTracking2017}.
\subsection{Vinculación con mecanismos de búsqueda en espacios naturales}
\label{sec:orga5d7bdf}

Finalmente, cómo objetivo a mayor largo plazo, y sólo a nivel exploratorio,
vincular las estructuras subyacentes a búsqueda semántica y forrajeo con sus
raíces evolutivas. De esta manera generalizando la función de dichas estructuras
al comportamiento orientado a objetivos \citep{hillsAnimalForagingEvolution2006} 
\section{Compromiso}
\label{sec:org8d1befe}
\subsection{Disposición de investigación, demostrar comportamiento pasado}
\label{sec:org4989e49}
\subsection{Disposición a aprendizaje autonomo detallando técnicas a aprender}
\label{sec:orgf1c737a}
\bibliographystyle{apalike}
\bibliography{ref}
\end{document}
